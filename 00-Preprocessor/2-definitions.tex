%!TEX root = ./main.tex
%%colores definidos%%
\definecolor{coolBlue}{RGB}{0, 84, 186}
\definecolor{rojoHierro}{RGB}{240, 79, 41}
\definecolor{verdeSapo}{RGB}{0, 153  , 76}
\definecolor{rojoASTM}{RGB}{139, 21  , 26}
\definecolor{ocre}{RGB}{243,102,25} % Define the orange color used for highlighting throughout the book
%%Definiciones para autoref%%
 \def\equationautorefname~#1\null{%
   ec.#1\null
 }

 \def\figureautorefname~#1\null{%
   fig.#1\null
 }
 
 \def\subsectionautorefname~#1\null{
    sec.#1\null
 }
 
  \def\tableautorefname~#1\null{
    tab.#1\null
 }
 
 \def\subsubsectionautorefname~#1\null{
    sec.#1\null
 }
 
  \def\sectionautorefname~#1\null{
    sec.#1\null
 }
 
 
%%Tipografía%%
\newcommand*\tick{\item[\ding{51}] } %genera un visto
\newcommand*\fail{\item[\ding{55}]} %genera una equis
\newcommand*\bull{\item[$\bullet$]} %genera balas

%%Definiciones para paths%%
\graphicspath{{09-Pictures/}} % Specifies the directory where pictures are stored
%\graphicspath{{10-Tables/}} % Specifies the directory where pictures are stored
 


% \sisetup{
% list-final-separator = { \translate{and} },
% list-pair-separator = { \translate{and} },
% range-phrase = { \translate{to (numerical range)} },
% }

% \selectlanguage{spanish}%

%%titulo
\newcommand\subject{Física Estadística}
\newcommand\topic{Deber 4}

\newcommand*\circled[1]{\tikz[baseline=(char.base)]{
            \node[shape=circle,draw,inner sep=2pt] (char) {#1};}}